\documentclass[12pt]{article}
\usepackage[utf8]{inputenc}
\usepackage[spanish]{babel}
\usepackage{amsmath}
\usepackage{graphicx}
\usepackage{booktabs}

\title{Reporte Técnico: Optimización de Procesamiento Agrícola}
\author{Nombre del Estudiante: \underline{\hspace{5cm}}}
\date{\today}

\begin{document}

\maketitle

\section{Introducción}
En este espacio, describa brevemente el problema de procesar grandes volúmenes de datos de sensores de humedad y por qué la eficiencia es crítica en la agroindustria.

\section{Resultados del Benchmark}
Complete la siguiente tabla con los tiempos obtenidos al ejecutar el script \texttt{benchmark.py} en su entorno local:

\begin{table}[h]
\centering
\begin{tabular}{lr}
\toprule
\textbf{Método} & \textbf{Tiempo de Ejecución (s)} \\
\midrule
Bucle For Tradicional &  \\
Vectorización NumPy &  \\
\bottomrule
\end{tabular}
\caption{Comparación de eficiencia medida por el estudiante.}
\end{table}

\textbf{Cálculo del Speedup:} 
NumPy fue aproximadamente \underline{\hspace{2cm}} veces más rápido que el bucle tradicional.

\section{Análisis de Implementación}
Explique brevemente qué funciones de NumPy utilizó para las tareas de:
\begin{enumerate}
    \item Limpieza de datos (Saneamiento).
    \item Identificación de zonas críticas (Máscaras).
    \item Simulación de riego (Lógica condicional).
\end{enumerate}

\section{Conclusiones}
Escriba sus conclusiones sobre el impacto de usar computación de alto rendimiento (HPC) en tareas de monitoreo agrícola.

\end{document}
